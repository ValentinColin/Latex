\documentclass[a4paper, french, twoside]{article}

% import des packages de base
\usepackage[latin1]{inputenc}	% pour le codage 
\usepackage[T1]{fontenc}		% pour la police (et donc les accents)
\usepackage[french]{babel}	% pour les commandes en fran�ais
\frenchbsetup{StandardLists=true} % � inclure si on utilise \usepackage[french]{babel}
\usepackage{textcomp}		% pour afficher correctement l'apostrophe
\usepackage{lmodern}		% Pour changer le pack de police
\usepackage[left=3cm, right=3cm, bottom=3cm]{geometry}	% pour personaliser les marges
\usepackage{graphicx}		% pour inclure des images
\usepackage{fancybox}		% pour les boites aux bords arrondie
\usepackage{amsmath}		% pour des commandes math�matiques (surtout l'alignement)
\usepackage{amssymb}		% pour la notation des ensembles math�matiques (ensembles des entiers naturels)
\usepackage{mathrsfs}		% pour les lettres calligraphiques anglaise







% En-t�te/Pieds de page
\usepackage{fancyhdr}
\pagestyle{fancy}
	% En-t�te
\renewcommand{\headrulewidth}{0.4pt}	% Trait de s�paration (mettre 0pt pour la retirer)
\fancyhead[L]{I.S.E.P 2019-2020}
\fancyhead[C]{Devoir libre n�5}
\fancyhead[R]{SP�}
	% Pieds de page
\renewcommand{\footrulewidth}{0.4pt}	% Trait de s�paration (mettre 0pt pour la retirer)
\fancyfoot[L]{M. Popot--Chapuis/ V. Colin / M. Partensky}
\fancyfoot[C]{}
\fancyfoot[R]{Page \thepage}



\begin{document}

\begin{center}
\huge
Devoir Maison n�5\\
\Huge
Autour du laplacien
\end{center}
\hrule


\part{Mise en bouche}
(partie r�dig�e par Maxime)

\part{Fonctions radiales harmoniques}


\begin{enumerate}
	\item 
        		\begin{enumerate}
            		\item (partie r�dig�e par Maxime)
            		\item (partie r�dig�e par Maxime)
        		\end{enumerate}
    	\item 
		\begin{enumerate}
            		\item Puisque la norme euclidienne ($\| x\| = \sqrt{x_1^2 + ... + x_k^2 + ... + x_{n}^2}$ o� $x_k$ est la $k$-i�me coordonn�e de $x$) est de classe $\mathscr{C}^2$ par rapport � chaque variable sur $\mathbb{R}_{+}^{*}$ et que $f$ est de classe $\mathscr{C}^2$ alors $\varphi$ est de classe $\mathscr{C}^2$ sur $\mathbb{R}_{+}^{*}$
            		\item {Notons $N$ la norme euclidienne. On a alors: \\
\[\forall x \in \mathbb{R}^n \quad \dfrac{\partial N}{\partial x_k}(x) = 2 x_k . \dfrac{1}{2}(x_1^2+...+x_n^2)^{-\frac{1}{2}} = \dfrac{x_k}{\sqrt{x_1^2+...+x_n^2}} = \dfrac{x_k}{\| x\|}\]
Ainsi
\[\dfrac{\partial f}{\partial x_k} = \dfrac{\partial (\varphi \circ N)}{\partial x_k} = (\varphi' \circ N).\dfrac{\partial N}{\partial x_k}\]}
			\item {En d�rivant le produit ci-dessus, on obtient:
\[\dfrac{\partial^2 f}{\partial x_k^2} =  (\varphi' \circ N).\dfrac{\partial^2 N}{\partial x_k^2} + \left((\varphi'' \circ N).\dfrac{\partial N}{\partial x_k}\right).\dfrac{\partial N}{\partial x_k}\]

D'o�
\[\dfrac{\partial^2 f}{\partial x_k^2} = (\varphi' \circ N).\dfrac{\partial^2 N}{\partial x_k^2} + (\varphi'' \circ N).\left(\dfrac{\partial N}{\partial x_k}\right)^2\]

Puisque le Laplacien s'�crit:
\[\Delta f = \sum_{k=1}^{n} \dfrac{\partial^2 f}{\partial x_k^2}\]

Alors
\[\Delta f =  (\varphi' \circ N).\sum_{k=1}^{n}\dfrac{\partial^2 N}{\partial x_k^2} + (\varphi'' \circ N).\sum_{k=1}^{n}\left(\dfrac{\partial N}{\partial x_k}\right)^2\]
On remarque bien �videmment la forme obtenue dans la premi�re partie
\[\Delta (\varphi \circ N) = (\varphi' \circ N).\Delta N + (\varphi'' \circ N).\| \nabla N\|^2 \]

\newpage
Calculons � pr�sent $\| \nabla N\|^2$ et $\Delta N$.\\
On a vu que \[\dfrac{\partial N}{\partial x_k}(x) = \dfrac{x_k}{\|x\|}\]
D'o�
\[\nabla N(x) = 
\begin{pmatrix}
\dfrac{x_1}{\|x\|} \\[2mm]
\vdots 		 \\[2mm]
\dfrac{x_n}{\|x\|} \\
\end{pmatrix}  \]

Ainsi
\begin{eqnarray*}
\|\nabla N(x)\|^2 &=& \left<\nabla N(x) | \nabla N(x)\right>\\
 			 &=& \sum_{k=1}^{n}\dfrac{x_k^2}{\|x\|^2}\\
 			 &=& \dfrac{1}{\|x\|^2}\sum_{k=1}^{n}x_k^2\\
		     	 &=& \dfrac{\|x\|^2}{\|x\|^2}\\
		     	 &=& 1
\end{eqnarray*} \\

Ensuite calculons $\Delta N$\\
On obtient d�j� que (d�river d'un quotient):
\[\forall x \in \mathbb{R}^n \quad \dfrac{\partial^2 N}{\partial x_k^2}(x) = \dfrac{1.N(x)-x_k.\frac{x_k}{N(x)}}{N(x)^2} = \dfrac{N(x)^2-x_k^2}{N(x)^3}\] 

Ainsi
\begin{eqnarray*}
\Delta N(x) &=& \sum_{k=1}^{n}\dfrac{\partial^2 N}{\partial x_k^2}(x)\\
 	      &=& \sum_{k=1}^{n}\dfrac{\|x\|^2-x_k^2}{\|x\|^3}\\
 	      &=& \dfrac{1}{\|x\|^3}\left(n\|x\|^2-\|x\|^2\right)\\
	      &=& \dfrac{n-1}{\|x\|}
\end{eqnarray*} \\

Finalement on a:
\[\forall x \in \mathbb{R}^n \quad \Delta f(x) = \varphi''(\|x\|) + \dfrac{n-1}{\|x\|}\varphi'(\|x\|) \]}
        		\end{enumerate}
		
\newpage
    	\item {Soit $f$ un fonction radiale harmonique et $x$ un vecteur de $\mathbb{R}^n$. On a donc $\Delta f = 0$.\\
Or d'apr�s la premi�re partie on connait la forme du laplacien d'une compos�e.
\[\Delta f(x) = \varphi''(\|x\|) + \dfrac{n-1}{\|x\|}\varphi'(\|x\|)\] \\
On a donc, en posant $r = \|x\|$ \[\varphi''(r) + \dfrac{n-1}{r}\varphi'(r) = 0\]
On remarque l'�criture d'un �quation diff�rentielle du second ordre sans terme non-d�riv�. On peut donc poser $\omega = \varphi'$. Ainsi on a
\[\omega'+ \dfrac{n-1}{r}\omega = 0 \]
$\omega(r) = \dfrac{1}{r^{n-1}}$ est une solution particuli�re de l'�quation diff�rentielle.\\
Ainsi \[ \varphi'(r) = \dfrac{\lambda}{r^{n-1}} \qquad\text{avec}\quad \lambda \in \mathbb{R}\]
En primitivant $\varphi'$ on obtient: \\
Par cas:
\begin{itemize}
	\item {Si n = 2:\\
Alors \[\varphi' = \dfrac{\lambda}{r}\]
Donc \[f(x) = \varphi(r) = \lambda.ln(r) + \mu \quad\text{avec $\lambda$ et $\mu$ des r�els}\]
}
	\item{Si n > 2:\\
Alors \[\varphi' = \dfrac{\lambda}{r^{n-1}}\]
Donc \[f(x) = \varphi(r) = \dfrac{\lambda}{(2-n).r^{n-2}} + \mu \quad\text{avec $\lambda$ et $\mu$ des r�els}\]
}
\end{itemize}

Conclusion: \\
Les fonctions radiales harmoniques sont:
\begin{equation*}
\boxed{f(x) = \lambda.ln(r) + \mu} \tag{n = 2}
\end{equation*}
Ou
\begin{equation*}
\boxed{f(x) = \dfrac{\lambda}{(2-n).r^{n-2}} + \mu} \tag{n > 2}
\end{equation*}


} % fin de la question 3)
\end{enumerate}


\part{Laplacien et isom�tries}

(partie r�dig�e par Marc)



\end{document}













